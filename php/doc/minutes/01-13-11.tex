\documentclass[12pt]{article}
\usepackage[utf8]{inputenc}
\usepackage{amsmath}
\title{GusPHP}
\date{January 13th, 2011}
\begin{document}
\maketitle
 
\textbf{Minutes of the 1st meeting of Spring Semester 2011} \\
Colby Blair - Group Leader\\
\textbf{McClure 123, University of Idaho Campus, Moscow ID 83843} \\
\textbf{Present}\\
Brett\\
Abhay\\
Zeke\\
Alex\\
Cindy\\
John\\
Chaylo\\
Colby\\
Scott\\
Tim\\
\textbf{In Attendance}\\
None \\
\section{Actions for consideration} \\
\subsection{Homework 1 Requirements}
 \begin{enumerate}
 \item Hold team meeting - done, in progress
 \item "...develop initial class files for ALL your team's classes...", "Put in "Doc"-style ... header comments for classes, class variables, and public methods",
 \begin{itemize}
  \item Identify all customer requirements. Comes from old class notes, last semester homeworks, and last semester SSRS's
  \item Controllers
  \begin{itemize}
   \item forum -- Chaylo
   \item DB inserts
   \item DB wrapper
  \item document sharing -- file management -- Alex
  \item email and messaging -- Abhay
  \item group and user management, dues, expenses -- Colby (sponsor), Leah (officer), Tim
  \item calendar - users and groups -- Zeke
  \item project manager -- time sheets, schedule
  \item Navigation -- Scott
  \item group and user pages -- Brett
  \item standard display - pulls from users and groups, controllers for user and group view.
  \item searching for existing user groups -- could insert google custom search -- Tim?
  \item security -- hacking site -- Zeke?
  \item documentation team -- Cynthia, Leah
  \item Google page UI GUS - PHP -- Brett
 \end{itemize}
 \item DB model
  \item Identify (non-requirements?) for ourselves in the project.
   \begin{itemize}
    \item Gus tools we require -- Code Igniter,
    \item Dividing the Labor -- minutes
    \item Scheduling delivery of various parts of the system
    \item Where Does the Buck Stop? Colby
    \item How do we document bugs? Trac, maybe github
    \item How do we evaluate the system? CI internal test framework
    \item How are requirements to be documented? document yourself and documentation team follows after and writes more class documentation
    Write code, do basic documentation -- class diagrams, and documentation team follows with user defined associations. Communicate with doc
    team to make sure your code matches their Use Cases and general design docs.
    \item Who talks to the client? We all will when we have our first iteration, focusing repectively on the clients that match our subprojects.
   \end{itemize}
  \end{itemize}
 \item "write either the actual code bodies of methods, or else write commented-out pseudocode, ... write code that calls real methods in other classes. Where necessary, write code that returns "fake" values..."
 \begin{itemize}
  \item The current sprint ends on Jan 26th
 \end{itemize}
 \item "produce written estimates of":
 \begin{itemize}
  \item "what percent of your requirements are currently covered by your design" - 100\%
  \item "what percent of your design is currently implemented by your preliminary code" - 5\%
  \item "what percent of your code is already tested by reproducible test cases." - 0\%
 \end{itemize}
\end{enumerate}
 \section{Any Other Business}
 \begin{itemize}
 \item Open floor
\end{itemize}
\section{New Business}
None\\
\textbf{Meeting closed at <time>} \\
Closed by:
Date
7:00 PM Tuesdays Thursdays Skype meetings
\end{document}

