\chapter{Introduction}
	\section{Identification}
		The software system being considered for development is referred
		to as Groups in a University Setting or gus. The customer providing
		specifications for the ethnic and religion team is the Lutheran
		Campus Ministry. The ultimate customer, or end-user, of the system
		will be student groups at the University of Idaho. This is a new
		project effort, so the version under development is version 0.05
	\section{Purpose}
		The purpose of the system under development is to provide a tool
		for the easy administration and tracking of university-style groups
		including but not limited to clubs and sports teams, the system will
		also try to increase student involvement by connecting and recognizing
		the involvement of users. While the system will be used by university
		personnel, this document is intended to be read and understood by UICS
		software designers and coders.  This document will also be approved by
		Dr. Clinton Jeffery.
	\section{Scope}
		GUS is to social-networking as an intranet is to the Internet.
		Where other social networks distract users with non-university
		non-local non-face-to-face non-involvement, GUS will focus these
		types of functionalities to the university and local community
		setting to best meet the administrative and service needs of groups
		in a university setting.  In addition to being a group-centered
		student-involvement web-application.
	\section{Definitions, Acronyms, and Abbreviations}
		\begin{tabular}{|p{4cm}|p{10cm}|}
		\hline
		\textbf{Term or Acronym} & \textbf{Definition} \\ \hline
		Alpha test & Limited release(s) to selected, outside testers \\ \hline
		Beta test & Limited release(s) to cooperating customers wanting early access to developing systems \\ \hline
		Final test & aka, Acceptance test, release of full functionality to customer for approval \\ \hline
		DFD & Data Flow Diagram \\ \hline
		SDD & Software Design Document, aka SDS, Software Design Specification \\ \hline
		SRS &  Software Requirements Specification \\ \hline
		SSRS & System and Software Requirements Specification \\ \hline
		GUS & Groups in a University Setting \\ \hline
		\end{tabular}
	\section{References}
		\begin{enumerate}
			\item www.churchteams.com
			\item www.groupmeister.com
			\item www.teamr.com
			\item www.salesboom.com
			\item www.wikipedia.org
		\end{enumerate}
	\section{Overview and Restrictions}

	This document is for limited release only to UI CS personnel
	working on the project.

		Section 2 of this document describes the system under development
		from a holistic point of view.  Functions, characteristics,
		constraints, assumptions, dependencies, and overall requirements
		are defined from the system-level perspective.

		Section 3 of this document describes the specific requirements of
		the system being developed.  Interfaces, features, and specific
		requirements are enumerated and described to a degree sufficient
		for a knowledgeable designer or coder to begin crafting an
		architectural solution to the proposed system.

		Section 4 provides the requirements traceability information for
		the project.  Each feature of the system is indexed by the SSRS
		requirement number and linked to its SDD and test references.

		Sections 5 and up are appendices including original information
		and communications used to create this document.

