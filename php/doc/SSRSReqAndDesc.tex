\chapter{Requirements Traceability}
\part{Systems and Software Requirements Specification (SSRS) \\ for \\ Groups in a University Setting}
\chapter{Overall Description}
	\section{Product Perspective}
		Gus is an independent software system, as it does not directly
		integrate with a larger system. However, GUS does draw data from
		external sources, such as personal information databases, and
		needs to be integrated with a web server in order to be readily
		accessible.
	\section{Product Functions}
		\begin{enumerate}
		\item Simplifying tasks to leaders of groups, such as:
			\begin{enumerate}
				\item Sending notifications to group members, prospective
					members, former members, and interested community
					members (email)
				\item Sending information (files) to group members via
					email or download link
				\item Managing a group-wide calendar of events
				\begin{enumerate}
					\item track volunteers, attendees, and contributors
					\item suggest potentially beneficial services other
						groups could provide related to the event
					\item provide a calendar of events that includes events
						from other groups that members would want to attend
						(like marching band if half of LCM are in the
						marching	band.)
					\item keeping track of who is responsible for bringing
						/ doing what at an event
				\end{enumerate}
				\item Automatically generating:
				\begin{enumerate}
					\item Contact information (contact sheets, phone
						directories)
					\item Website with updated contact, group, event, and
						customized information
					\item Organization charts
					\item Graphical relationships between groups
					\item Fees, dues, and expenses notifications
					\item Event reminders
				\end{enumerate}
			\end{enumerate}
			\item Consolidating information for members, former members,
				potential members (and parents) of groups:
			\begin{enumerate}
				\item Common location of group information
				\item Searching existing groups
				\item Tying together existing groups (even suggesting
					similar groups)
				\item Personalized emails regarding changes/updates
				\item Outstanding expenses or reimbursements
				\item Reliable (i.e., automatically updated):
				\begin{enumerate}
					\item Group contact information
					\item Group event information
				\end{enumerate}
				\item Transcript of verified group activity (for use with
					service-learning classes, and proof of volunteerism
					for potential employers)
				\item Supplementing Vandal Friday with emails to
					prospective high school Seniors
			\end{enumerate}
			\item Getting member input though: forums, project managers,
				surveys, and polls
			\item Payment processing and sponsorship collection
			\item 4. Recruitment and advertising for groups, volunteer
				/ paid opportunities, services provided, possibly a
				bartering tool
		\end{enumerate}
	\section{User Characteristics}
		Gus should be easy for any user to understand with a brief
		explanation and intuitive enough for an uninitiated user to
		figure out by looking through the options. Basic computer use
		skills and a simple conceptual explanation should be enough for
		every day usage.
	\section{Constraints}
		GUS must meet privacy policies as they apply to both the
		University of Idaho and social networking sites.  GUS must be
		able to interface with outside database servers (such as the
		Center for Volunteerism's database, UI's career seeker site,
		common social networks, and parent groups of university groups).
		Member's activities and group's activities must be audited for
		accuracy and safety.  The languages used to program GUS will be
		primarily, HTML, CSS, and PHP for the user interface, C++ for
		the interface between the user interface (which will implement
		security and complex business rules), and the database, and
		SQL for the database.  The networking protocols will be TCP/IP
		and Open MP / MPI will be used to enhance parallel operation.
		The system will have personal information for over 5,000
		students, so confidentiality is of the utmost importance.
	\section{Assumptions and Dependencies}
		The software system should run like a web-app and need not be
		downloaded by users.  It is assumed that users will be running
		Internet Explorer, Fire Fox, or another popular web browser.
		The server for the system is expected to run a UNIX operating
		system.
	\section{System Level (Non-Functional) Requirements}
		\subsection{Site dependencies}
			GUS will require a server that can support 1,000 concurrent
			users.  The database must store the information, interests,
			and activities of approximately 5,000 external users, 5,000
			students and 200 groups.
		\subsection{Safety, security and privacy requirements}
			GUS contains the personal information of over 5,000 users
			security should be integrated into every facet of this
			program.  The privacy criteria for this system must reflect
			privacy policies that apply to the University of Idaho, and
			the security criteria for this system must reflect the need
			to secure over 5,000 users from identity theft and potential
			defamation of character.
		\subsection{Performance requirements}
		\begin{enumerate}
			\item The number of simultaneous users to be supported are: 1,000.
			\item Supported information ranges from text to files to streaming video.
			\item 95\% of the transactions shall be processed in less than half a second.
		\end{enumerate}
		\subsection{System and software quality}
			Gus must perform all required functions, behave consistently
			and correctly, be easily corrected, running between 5:30 am
			all day to 1:30 am be easily adaptable, test-driven, and easy
			to use.
		\subsection{Packaging and delivery requirements}
			The executable system and all associated documentation (i.e.,
			SSRS, SDD, code listing, test plan (data and results), and
			user manual) will be delivered to the customer via Internet
			download. The final, edited version of the above documents
			will accompany the final, accepted version of the executable system.
		\subsection{Personnel-related requirements}
			The system under development will require a graduate student
			system-level administrator to maintain the system.
		\subsection{Training-related requirements}
			No training materials or expectations are tied to this
			project other than the limited help screens built into the
			software and the accompanying user manual.
		\subsection{Logistics-related requirements}
			A server will be required to maintain the software system.
			The user will be required to have an Internet connection.
		\subsection{Precedence and criticality of requirements}
			\begin{enumerate}
				\item Maintaining confidentiality and privacy of PII
				\item This system must be reliable enough for users to not
					give up on it
				\item All other features are less important than the first
					two and equally important
			\end{enumerate}

