%\documentclass[12pt, twoside, letterpaper]{proc}
\documentclass[12pt, twoside, letterpaper]{report}
\usepackage{ifpdf}
\usepackage[colorlinks,bookmarksopen]{hyperref}
\usepackage{graphicx}
\usepackage{fullpage}
\usepackage{picinpar}
\usepackage{wrapfig}
\begin{document}
\section{Database}

\subsection{The Users Class}
The users table represents the users, who form groups, own
pages, make posts, and attend events.
\begin{wrapfigure}[7]{r}{2.5in}
\begin{center}
\caption{UML class description}
\includegraphics{tex/tbl_user_describe.png}
\end{center}
\end{wrapfigure}
\begin{itemize}
\item[u\_id] This uniquely identifies the user
\item[f\_name] The user's first name, to display on
     membership rosters
\item[l\_name] The user's last name, to display on
     membership rosters
\item[user\_name] This uniquely identifies the user, to
     display on pages and posts
\item[phone] The user's phone number, to send text-messages,
     and display on membership rosters
\end{itemize}

\begin{itemize}
\item[email] The email to which notifications and
     invitations are sent
\item[pass\_word] A 20 character array to hold the user's
     hashed pass word
\item[hash\_str] A 20 character array for the hash used to
     encrypt the user's password
\end{itemize}
\begin{figure}[h]
\begin{center}
\caption{Sample contents of the users table}
\includegraphics[width=6.5in]{tex/tbl_user_sample.png}
\end{center}
\end{figure}
\pagebreak
\subsection{The Groups Class}
The groups table represents the groups, that have
    membership rosters, own pages, host forums, and host
    events.\\
\begin{wrapfigure}[8]{r}{2.5in}
\begin{center}
\caption{UML class description}
\includegraphics{tex/tbl_group_describe.png}
\end{center}
\end{wrapfigure}

\begin{itemize}
\item[g\_id] This uniquely identifies the group
\item[g\_name] The name of the group
\item[acronym] The group's acronym
\item[category] The group's category: academic, honorary,
     community\_service, civic\_action, cultural, ethnic,
     graduate, professional, greek\_life, career,
     recreational, sport, special\_interest, spiritual, or
     religious
\item[email] The email users outside the group use to get
     in contact with the group
\end{itemize}
\begin{figure}[h]
\begin{center}
\caption{Sample contents of the groups table}
\includegraphics[width=6.5in]{tex/tbl_group_sample.png}
\end{center}
\end{figure}
\pagebreak
\subsection{The Membership Associative Table}
Each row in the membership table represents a member, a
group they are in, and the status of the member.\\
\begin{wrapfigure}[6]{r}{2.5in}
\begin{center}
\caption{UML class description}
\includegraphics{tex/tbl_membership_describe.png}
\end{center}
\end{wrapfigure}
\begin{itemize}
\item[u\_id] The ID of the user
\item[g\_id] The ID of the group
\item[status] The status of the user within the group,
     limited to common statuses such as, member, officer,
     and sponsor.  The database-wide roles (ASUI, ITS) will
     be implemented through GRANT statements.
\end{itemize}

\begin{wrapfigure}[9]{r}{2.5in}
\begin{center}
\caption{Sample contents of the membership table}
\includegraphics[width=2in]{tex/tbl_membership_sample.png}
\end{center}
\end{wrapfigure}
The ID numbers where used to speed the query transactions,
but through the use of the appropriate query, it's possible
to get more human-readable results.\\
A straight transliteration of the membership table might
include the user-name, the group name, and the membership
status. I would need to use information from three tables:
the user table, the group table, and the membership table.
Then I would make sure I only got the entries where the
u\_ids matched, and the g\_ids matched.
SELECT u.user\_name, g.g\_name, m.status\\
FROM gus.tbl\_membership m\\
    INNER JOIN gus.tbl\_group g\\
    INNER JOIN gus.tbl\_user u\\
      ON g.g\_id=m.g\_id AND u.u\_id=m.u\_id;\\

\begin{figure}[h]
\begin{center}
\caption{Human-readable version of the membership table}
\includegraphics[width=6.5in]{tex/qry_membership_sample.png}
\end{center}
\end{figure}
\end{document}
