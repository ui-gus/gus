\subsection{Requirements Traceability}
In the case of GUSPHP, all requirements originated from the customer, Dr. Jeffery. Though extensive discussion both with Dr. Jeffery and conversations as a team, a general set of requirements were eventually agreed upon. These requirements formed the core goals that GUS strives for. \\
\\
Once a set of requirements had been agreed upon, actual implementation details were then confered upon as a team. These details included use-case creation and division of labor. Since GUS is a web-based project, each requirement formed a different segment of functionality of the overall project. With this being the case, each team member is assigned a specific requirement. Their responsibility is to ensure that their part was completed by the end of the semester.
\\
Development can be traced by tickets present on the GUS Trac server. Trac is an online project managment tool that easily shows team progress through a project roadmap. When properly updated, the roadmap shows a detailed view of the development cycle by utilizing a ticket system. The GUSPHP team leader will add tickets - or small segments of work - for each team member. These tickets will then provide an work log for each employee, as well as reflecting progress towards a complete implementation. 
